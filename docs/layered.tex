\documentclass[a4paper,10pt]{paper}
\usepackage[utf8]{inputenc}
\usepackage{graphicx}
\usepackage{amsmath}

\setlength\textwidth{6.5in}
\setlength\oddsidemargin{0in}
\setlength\evensidemargin{0in}

\begin{document}
\maketitle

\section{Layered View: Client}

\subsection{Primary presentation}

\begin{center}
    \begin{figure}
      \includegraphics[width=\textwidth]{../images/layered_client.jpg}
    \end{figure}
  \end{center}

\subsection{Element catalog}

\subsubsection{Elements and their properties}

-Software Layer: naam, verantwoordelijkheid , inhoud\\

Client UI Layer\\

Verantwoordelijkheid:\\
Uiterlijk presentatie van Client Applications.\\

Inhoud:\\
GUI (buiten project scope)\\

Client Application Layer\\

Verantwoordelijkheid:\\
-Implementatie van specifieke business logica.\\
-Validatie van gebruikers identiteit. Enkel bevoegde personene kunnen bepaalde applicatie gebruiken en toegang krijgen tot bijhorende data.\\

Inhoud:\\
GP Application (buiten project scope)\\
Card Reader Module\\

Client Data Layer\\

Verantwoordelijkheid:\\
-Uniformaliteit aanbieden voor de toegang tot data's. In normaal geval moet de gebruiker niet weten waar de data vandaan gehaald wordt, lokale cache of external data server.\\
-Standaard v��r- en nabewerkingen uitvoeren op data's: encryptie/decryptie, compressie/decompressie.\\
-Opslaan van lokale data cache en consistentie onderhouden tussen lokale cache en external data server.\\

Inhoud:\\
Data Manager, Cache, Security (Authentication en Compression)\\

Client Communication Layer\\

Verantwoordelijkheid:\\
Communicatie met externe systemen (GMR Server).\\

Inhoud:\\
Network Module (buiten project scope)\\


\subsubsection{Relations and their properties}

-Allowed-to-Use: Layer A en Layer B hebben de relatie "A is allowed-to-use B" als A onmiddellijk boven B staat, tenzijn anders expliiciet vermeld.\\

\subsubsection{Element interfaces}

\subsubsection{Element behavior}

\subsection{Context diagram}

\subsection{Variability guide}

\subsection{Architecture background}

\subsubsection{Rationale}

\subsubsection{Analysis results}

\subsubsection{Assumptions}

\subsection{Other information}

\subsection{Related view packets}

Sibling: Layered View: Server\\
Child: Uses View: Client\\


\section{Layered View: Server}

\subsection{Primary presentation}
\begin{center}
    \begin{figure}
      \includegraphics[width=\textwidth]{../images/layered_server.jpg}
    \end{figure}
  \end{center}

\subsection{Element catalog}

\subsubsection{Elements and their properties}

-Layer: naam, verantwoordelijkheid , inhoud\\

Server Interface Layer\\

Verantwoordelijkheid:\\
Verbindingen van gebruikers ontvangen en delegeren naar Public Services Layer.\\

Inhoud:\\
Proxy\\

Public Services Layer\\

Verantwoordelijkheid:\\
Publiek service ter beschikking stellen voor gebruikers. Aan de hand van gekozen publiek service worden de onderliggende kern service opgeroepen.\\
Op dit niveau wordt zijn de gebruikers nog volledige anoniem.\\

Inhoud:\\
Public Service\\

Core Services Layer\\

Verantwoordelijkheid:\\
Implementatie van business logica van eHealth platform.\\
Basis bouwblokken voorzien voor publoek service.\\
Identificatie van gebruiker en de bevoegdheid van gebruiker controleren.\\

Inhoud:\\
GMR, Security, Session Management, Policy\\

Virtual Data Layer\\

Verantwoordelijkheid:\\
Bescherming van data's tegen onbevoegde toegang.\\
Uniformaliteit aanbieden voor de toegang tot data's.\\

Inhoud:\\
Virtual GMR Data\\

Fysic Data Layer\\

Verantwoordelijkheid:\\
Bescherming van data's tegen data-corruptie.\\
Opslag van Medical Records.\\
Opslag van Security data's die nodig zijn voor Authorization/Authentication van client gebruikers.\\

Inhoud:\\
GMR Data Server, Doctor Data Server, Security Server\\


Server Communication Layer\\

Verantwoordelijkheid:\\
Communicatie met externe systemen (RIZIZ, Log).\\

Inhoud:\\
Network Module (buiten project scope)\\




\subsubsection{Relations and their properties}

-Allowed-to-Use: Layer A en Layer B hebben de relatie "A is allowed-to-use B" als A onmiddellijk boven B staat, tenzijn anders expliiciet vermeld.\\

\subsubsection{Element interfaces}

\subsubsection{Element behavior}

\subsection{Context diagram}

\subsection{Variability guide}

\subsection{Architecture background}

\subsubsection{Rationale}

\subsubsection{Analysis results}

\subsubsection{Assumptions}

\subsection{Other information}

\subsection{Related view packets}

Sibling: Layered View: Client\\
Child: Uses View: Server\\

\section{Uses View: Client}

\subsection{Primary presentation}
\begin{center}
    \begin{figure}
      \includegraphics[width=\textwidth]{../images/uses_client.jpg}
    \end{figure}
  \end{center}

\subsection{Element catalog}

\subsubsection{Elements and their properties}

-Software Module: naam, beschrijving\\

GUI\\
Bebruikers interface\\
(buiten project scope)\\

GP application\\
Professionele dokter software applicaties.\\
(buiten project scope)\\

Card Reader Module\\
Besturingssoftware voor Card Reader. Software interface van de hardware, maakt het mogelijk om de Card Reader te gebruiken door andere software applicaties.\\

Data Manager\\
Algemeen data management, camoufleert de exacte lokatie van data's, behandelt basis data bewerkingen en zorgt voor consistentie tussen lokale cache en external server data's.\\

Cache\\
zie C&C View: Client\\

Security\\
zie C&C View: Client\\

Compression\\
zie C&C View: Client\\

Network Module\\
Standaard network software applicaties.\\
(buiten project scope)\\


\subsubsection{Relations and their properties}

- Uses: een layer of een software module will gebruik maken van een andere layer of software module als de eerste expliciet verbonden is met de tweed. Wanneer de tweede entieit een layer is, mogen alle ingesloten software modules gebruikt worden door de eerste entiteit.\\

\subsubsection{Element interfaces}

\subsubsection{Element behavior}

\subsection{Context diagram}

\subsection{Variability guide}

\subsection{Architecture background}

\subsubsection{Rationale}

\subsubsection{Analysis results}

\subsubsection{Assumptions}

\subsection{Other information}

\subsection{Related view packets}

Sibling: Uses View: Server\\
Parent: Layered View: Client\\

\section{Uses View: Server}

\subsection{Primary presentation}
\begin{center}
    \begin{figure}
      \includegraphics[width=\textwidth]{../images/uses_server.jpg}
    \end{figure}
  \end{center}

\subsection{Element catalog}

\subsubsection{Elements and their properties}

Proxy\\
zie Deployment View\\

Public Service\\
zie Deployment View\\

GMR\\
zie C&C View\\

Security\\
zie C&C View\\

Policy\\
zie C&C View\\

Session Manager\\
zie C&C View\\

Virtuam GMR Data\\
zie Deployment View\\

Virtual GMR Data\\
zie Deployment View\\

Security Server\\
zie Deployment View\\

Network Module\\
Standaard network software applicaties.\\
(buiten project scope)\\


\subsubsection{Relations and their properties}

- Uses: een layer of een software module will gebruik maken van een andere layer of software module als de eerste expliciet verbonden is met de tweed. Wanneer de tweede entieit een layer is, mogen alle ingesloten software modules gebruikt worden door de eerste entiteit.\\

\subsubsection{Element interfaces}

\subsubsection{Element behavior}

\subsection{Context diagram}

\subsection{Variability guide}

\subsection{Architecture background}

\subsubsection{Rationale}

\subsubsection{Analysis results}

\subsubsection{Assumptions}

\subsection{Other information}

\subsection{Related view packets}

Sibling: Uses View: Client\\
Parent: Layered View: Server\\

\end{document}       